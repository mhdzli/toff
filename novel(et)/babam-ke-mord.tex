\documentclass[12pt,a4paper]{book}
\usepackage{hyperref}
\usepackage{epsfig,graphicx,subfigure,amsthm,amsmath}
\usepackage{color,xcolor}     
\usepackage{xepersian}
\usepackage[T1]{fontenc}
\usepackage[utf8]{inputenc}
\usepackage{lmodern}

\settextfont[Scale=1.2]{BZAR.TTF}
\setlatintextfont[Scale=1]{Times New Roman}

%%%%%%%%%%%%%%%%%%%%%%%%%%%%%%%%%%%%%%%%%%%%%%%%
% Chapter quote at the start of chapter        %
% Source: http://tex.stackexchange.com/a/53380 %
%%%%%%%%%%%%%%%%%%%%%%%%%%%%%%%%%%%%%%%%%%%%%%%%
\makeatletter
\renewcommand{\@chapapp}{}% Not necessary...
\newenvironment{chapquote}[2][2em]
  {\setlength{\@tempdima}{#1}%
   \def\chapquote@author{#2}%
   \parshape 1 \@tempdima \dimexpr\textwidth-2\@tempdima\relax%
   \itshape}
  {\par\normalfont\hfill--\ \chapquote@author\hspace*{\@tempdima}\par\bigskip}
\makeatother

\begin{document}

\title{بابام که مرد} 
\author{سید عباس سیاحی}
\date{تهران ۱۳۵۳}

\maketitle

\chapter*{مقدمه} 
\begin{chapquote}{علی اصغر مهاجر، \textit{بابام که مرد، شرکت سهامی کتاب‌های جیبی}}


«وقتی که آدم متمارض به بیماری روشنفکری نباشد، ساده حرف می‌زند. ولو آنکه حرفش مربوط باشد به مشکل‌ترین حالات روحی. سیاحی آدمی است ساده و سالم که از دهات حاشیه کویر آمده و در دهات و اطراف شهرها همراه ابوی ساده‌تر از خودش و به کمک تنها سرمایه‌شان که یک الاغ بارکش بوده نمک و تره‌بار فروخته و ناچار پای پیاده پهنه بزرگی از سرزمین عزیزمان را گز کرده است و همه جور مردمی را در اطراف و اکناف مملکت دیده است. بعد که به نوجوانی رسیده، زده به کله‌اش که درس بخواند و با اینکه خیلی دیر به این فکر افتاده خیلی خوب درس خوانده و شده است لیسانسه علوم تربیتی و بعد معلمی و کتاب نویسی. الان حدود بیست سالی هست که کتاب درسی می‌نویسد   برای شاگردان مدارس و برای بی‌سوادان بزرگسال. ده سالی هست که هر بچه اول ابتدایی در هر جای کشور که کتابش را باز کرده نام سیاحی را دیده و هنوز هم می‌بیند. اما کار سیاحی فقط تالیف کتاب نبوده است. ازهمان زمان‌ها که توی کوچه پس کوچه‌های شهرها با الاغ و بار نمکش آرام آرام راه می‌رفته و فریاد می‌زده: «آی نمکی»، حس کرده که ذاتا معلم است و تا عمر دارد باید معلمی بکند. معلم بشود و زیر سقف کلاس‌ها با صدای غرا درس بدهد و تربیت کند. همین کار را هم کرده و خدا می‌داند چند هزار بچه مدرسه را درس داده و به چند هزار معلم یاد داده که چگونه زبان فارسی را به بچه‌ها یاد بدهند.

و حالا این معلم ساده دهاتی داستان کوچکی نوشته که ساده‌تر از کتاب فارسی اول ابتدایی خوانده می‌شود، اما تصویر مردی را نشان می‌دهد که دوست ندارد دور از کوچه باغ‌های دهات معروض حوادث دردناک محتوم قرار گیرد و مردم شهر او را لخت و عور در برابر این حوادث تماشا کنند.»
\end{chapquote}

\tableofcontents

\chapter{دوشنبه ساعت ۱۱ بعد از ظهر}
امروز به دیدن سید یاسین رفتم. آفتاب غروب بود که آنجا رسیدم. سید یاسین روی پشت بام خانه مشغول زیر و رو کردن یونجه‌هایی بود که برای زمستان خرش خشک می‌کرد. من رفتم روی پشت بام. پشتش به من بود و توی کوچه را نگاه می‌کرد. با خواهرم که به او خبر داده بود که «رضا» آمد، حرف می‌زد.

سلام کردم. سرش را برگرداند که جوابم بدهد. از دیدن رنگ و رو و صورت گار کشیده‌اش وحشت کردم. خشکم زد. نزدیک بود فریاد بکشم که \textbf{چرا؟} چرا این طور شده‌ای؟ اما یک دفعه به فکرم رسید که نباید بترسانمش. این بود که باهاش دست دادم و روبوسی کردم و شروع به حرف زدن کردیم. روی همان پشت بام کاهگلی.

مثل اینکه او هم فهمید که من از حالتش ناراحت شدم. گفت پریروز نزدیک بود راحت بشم.

ضمن اینکه دلداری اش دادم گفتم: آخه شما که مسافرت براتون خوب نیست چرا رفتید مشهد؟

- اتفاقا مسافرت برام خیلی هم خوب بود اما پریروز که اومدم رفتم رد نمک، تو راه یک هندونه خریدم و خوردم، وقتی هم که اومدم خونه، مادرت استانبولی پخته بود، پلو  را که خوردم دلم درد گرفت و القصه نزدیک بود خلاص بشم.

- نه آقا خدا نکنه، حالا بیاید بریم پیش یک دکتر خوب.

- نه باباجون همین دکتر خیلی خوبه! دکتر سه تومنیه! سه تومن هم پول دواش می‌شه، روی هم شیش تومن، آدم خیلی هم زود خوب می‌شه.

- بسیار خوب هر طوری خودتون می‌دونید.
(هر دو روی زمین کاهگلی پشت بام نشستیم.)

سید یاسین از مسافرتش، از تجربیات زندگی‌اش و از تیرهایی که (به قول خودش) در زندگی توی شاخش خورده بود خیلی چیزها برایم گفت.
سید یاسین هفتاد و پنج سال دارد و به قول خودش سی سال است که عزراییل او را فراموش کرده است. اصلا خدا یک ذره ترس تو دل این مرد نگذاشته است.

سر پیری برای اینکه پشت سر مجتهد مرجع تقلیدش نماز بخواند و حضرت امام رضا را زیارت کند با صد و دو تومان پول بلند شده و رفته است به مشهد. تازه از این پول پنجاه تومانش را پول ماشین داده و سی تومانش را هم برگردانده، یعنی در عرض پانزده روز خرج خورد و خوراک این آدم فقط بیست و دو تومان بوده ولی با این همه خوش است که توانسته جلو جمعیت ده هزار نفری توی حرم شاه رضا «آقا» را دعا بکند.

امشب سید یاسین برای من خیلی نکته‌ها گفت:

\begin{itemize}

\item از عبادت‌های گذشته باید توبه کرد. خیلی از نماز شب‌هایی را که خوندم و خیال می‌کردم عبادت خدا را کرده‌ام کفر محض بوده. عبادت یعنی اینکه آدم چشم و گوشش را باز کنه و با چشم و گوش باز کار بکنه و کلاه سر کسی نگذاره و مواظب باشه که کسی هم کلاه سرش نگذاره.

\item ایاز را خواستند پیش سلطان محمود بدش کنند. یک روز که دستش را جلو دهنش گرفته و با سلطان حرف زده بود به سلطان گفتند: «ایاز می‌گه که چون دهن سلطان بوی بد می‌ده من دستم را جلو دهن و دماغم می‌گیرم که بوش ناراحتم نکنه.» سلطان محمود غضبناک شد و به ایاز گفت : «چرا دستت را جلو دهنت گرفتی و با من حرف زدی؟» ایاز گفت: «قربونت برم دوستان سیر توی غذا ریختند و بی‌خبر به خوردم دادند. هر کاری که کردم بوی سیر از دهنم نرفت و چاره‌ای هم نداشتم جز اینکه خدمت برسم و این بود که برای اینکه سلطان ناراحت نشوند دستم را جلو دهنم گرفتم و حرف زدم.»

\item یک روز رفتند پیش سلطان محمود و گفتند: «ایاز هر روز میره توی یک اتاق و سکه‌هایی را که از خزانه بلند کرده توی آن قایم میکنه و برای خودش مشغول دفینه درست کردنه.»

سلطان محمود ایاز را خواست و گفت: «تو اون اتاق هر روز چی‌کار داری؟» ایاز گفت: «بله عالم، قربون خاک پات برم استدعا می‌کنم خودتون تشریف بیارید و ببینید.»

همین که سلطان وارد اتاق شد یک دست لباس کهنه دید که به دیوار اتاق آویزونه. پرسید: «این لباس کهنه چیه؟» ایاز گفت: «قبل از این‌که مورد لطف و توجه قبله عالم قرار بگیرم دارایی من همین لباس کهنه بود. حالا هر روز میام و به این لباس کهنه نگاه می‌کنم تا یادم نره قبلا کی بودم.»

\end{itemize}

\chapter{شنبه دو ماه بعد}

امروز هم به دیدن سید یاسین رفتم. اما نه در خانه، در بیمارستان. من و دکتر عبادی با هم وارد اتاق شدیم. توی اتاق ده تا تخت گذاشته بودند و روی هر تخت یک مریض خوابانده بودند. آن گوشه اتاق یک دکتر و ده بیست تا از دانشجویان دانشکده پزشکی دور یک مریض جمع شده بودند. درست مثل لاشخورهایی که دور لاشه یک مردار جمع می‌شوند. بقیه مریض‌ها هم وحشت زده درست مثل گوسفندهایی که توی قصابخانه به انتظار نوبت سر بریدن خودشان گوسفندهای دیگر را نگاه می‌کنند، آقا دکتر و شاگردانش را تماشا می‌کردند و می‌خواستند یک جوری از مرض زندگی سوز هم اتاقیشان سر در بیاورند. بابام تختش وسط‌های ردیف شمالی اتاق بود. داشت «مریض گز»‌ها را تماشا می‌کرد. این اسمی است که خودش روی دانشجوهای پزشکی گذاشته بود. برگشت و مثل اینکه قبلا خبرش کرده باشند که به دیدنش رفته‌ام دست‌هایش را به طرف من که به طرفش می‌رفتم پیش آورد و گفت: «بابا الهی قربوند بشم چرا اینقدر زحمت می‌کشی؟» من از حالت بابام گریه‌ام گرفت اما باز هم جلو خودم را گرفتم.
خیلی وحشتناک بود. توی قیافه‌اش فقط وحشت دیدم.
وحشت مرگ. چشم‌هاش گود افتاده بود. روی استخوان‌هایش فقط یک پوست مانده بود. همان یک ذره گوشتی هم که در بدن لاغرش بود آب شده بود. ساق پاهاش درست به کلفتی مچ دست یک بچه شیری شده بود.

یاد گذشته بابام افتادم. برای من این پاها خیلی معنی دارد. این یک جفت پا با من خیلی چیزها می‌گویند. بابام با این پاها سه چهار بار پیاده به کربلا رفته. همین پاها هفت هشت بار به مشهد رفته.

این پاها همان پاهایی است که اقلا شصت سال آزگار روزی بیست تا سی کیلومتر راه رفته. این پاها همان پاهایی است که سر جاده طرق کار کرده و صد تومن پول نقره که یک من سنگ شاه وزن داشته، مزد گرفته و به خانه آورده است تا من که هنوز توی شکم مادرم بوده‌ام زنده بمانم و زندگی کنم. این پاها همان پاهایی است که تا یک ماه پیش از این به دنبال بار نمک می‌دویده تا مادر بد زبان و بی عاطفه من روزی ده دوازده تومن پول داشته باشد که خرج کند و به این پیرمرد بد و بیراه بگوید. درست است که این هم جوابش را می‌داد ولی من می‌دانم که آن زن چقدر این مرد را رنج می‌داد.

من شاهد زحمت کشیدن و خون جگر خوردن‌های این مرد بوده‌ام. من تمام دشت‌ها و بیابان‌های کاشان و قم و ساوه و زرند و غار و پشاپویه و شهریار و کرج و ورامین و ... را به همراه این مرد دویده‌ام و رنج‌هایی را که کشیده است دیده‌ام. من شاهدم که این مرد با روزی سی شاهی شکم خود را سیر می‌کرده است. در صورتی که همان روز ده دوازده تومان کاسب بوده و بقیه اش را برای آن زن فرستاده است. اما امروز توی بیمارستان حس کردم که همین پاها شاید تا یک ماه دیگر زیر خاک باشد.

بله «سید یاسین مظلومی» نمکی سر پل سیمان، که پدر من است کبدش از کارافتاده است. شکمش آب آورده است. شکمی که هرگز رنگ آبجو و عرق و شراب و ویسکی را ندیده، شکمی که اقلا شصت سال از هفتاد و پنج سال عمر صاحبش، با روزی یک نان جو خشک و خالی سیر شده است. من چه می‌دانم شاید هم حالا آب آورده است تا نان‌های خشک جو آب بکشند و خیس بخورند.

بابام امروز خیلی خوشحال بود. می‌گفت یک کیلو از وزنش کم شده است. یعنی یک کیلو آب شکمش در رفته است. بیچاره پیرمرد خودش نمی‌دانست و نمی‌دید که این یک کیلو آب شکمش نبوده بلکه ته مانده گوشت بدنش بوده است.

پیش از اینکه وارد اتاق بشویم با دکتر عبادی حرف زده بودم. راست یا دروغ، می‌گفت با این دواهایی که از بیرون خریدیم و به او می‌دهند حالش رو به بهبود است. گفت وضع پرستاری و غذای مریضخانه اصلا خوب نیست. پولی را که آدم بابت دوا می‌دهد فی‌الواقع دور می‌ریزد. دوا باید به قاعده و به موقع به مریض برسد. و کی برساند؟ پرستار. قرار شد فردا بابام را به خانه ببریم و بعد از ظهرش من بروم پیش دکتر و او شاید یک بیمارستان خصوصی پیدا کند که ارزان بگیرند و بعد بابایم را ببریم مریضخانه خصوصی. دکتر می‌گفت که اگر پرستاری خوبی ازش بکنند حالش خوب می‌شود.

به بابام گفتم: «فردا از اینجا می‌بریمتون» گفت: «نه همین جا خبه» بعد گفت: « آخه باباجون، قربونت بشم الهی، چقدر پول خرج کنی؟ همین جا می‌مونم، یا می‌میرم یا خوب می‌شم. همین جام که هستم روزی کلی پول دوا می‌دی. خیر سرشون تازه مثلا مریضخونه دولتیه. مگه من چی چی خرج تو کردم بابا که تو اینقدر خرج من می‌کنی؟»

گفتم: «نه آقاجون فکر این چیزهاش را نکن خدا خودش می‌رسونه. مگر اون سال را یادت نیست که از قم تا تهرون پیاده اومدیم و فقط نون خشک خریدیم و تو راه خوردیم. دیگه بدتر از اون که نمی‌شه. اگر هم نخوای مریضخونه خصوصی بریم باز خونه بهتره.»

همین که اسم خانه را شنید از وحشت رعشه به بدنش افتاد، گفت: « نه نه  همین مریضخونه خیلی به از خونست. الان یک کیلو شکمم وچیک شده.»

من می‌دانستم که برای پولش نیست. برای بهتر بودن بیمارستان هم نیست که می‌خواهد اینجا بماند. او از شنیدن کلمه «خونه» یاد فحش‌ها و نحسی‌های مادرم افتاد و وحشت کرد. این زن بد زبان بی عاطفه خیلی این مرد نازنین را چزانده.

بیچاره پدرم اگر وضع زندگی نکبت بار من را بفهمد آنا دق می‌کند. اگر مادرم بعد از چهل سال زندگی مثلا از دست بابام خسته شده و نحسی می‌کند، زن خودم بعد از هفت سال بنای فحش و بد و بیراه را گذاشته. تازه مادرم یک زن عامی و بی سواد است. اما زن من خدا نخواسته دیپلمه و تحصیل کرده است. تمام سعی این زن این است که من را از زندگی مایوس کند. هر کاری که من بکنم یک اما روش می‌گذارد. نشد که یک بار با هم بیرون برویم و از من صد تا ایراد بنی‌اسراییلی نگیرد. روزهای اول ازدواجمان همین که با هم پایمان را از در خانه بیرون می‌گذاشتیم، با سقلمه توی پهلوم می‌زد که: «چرا می‌شلی و راه می‌ری؟» من تا آن وقت نمی‌دانستم که می‌شلم. کسی هم به من نگفته بود که می‌شلی. زنم بری اولین بار این را به من گفت و از آن روزبه بعد راستی راستی می شلم و راه می‌روم. زنم معتقد است که برای اینکه لج او را در بیارم بیشتر می‌شلم. اما حقیقتش این است که از بس این زن به من تلقین کرده است، این طوری شده‌ام. از این موضوع گذشته همیشه زندگی فقیرانه پدر و مادرم و گذشته خودم را به رخم می‌کشد. از تربیت غلط خانوادگی من حرف می‌زند. تا دهن باز کنم خواهر و مادرم را جلو چشمم می‌جنباند. همیشه جلو بچه‌هام خوارم می‌کند. خیال می‌کند ازش می‌ترسم. هرچه ملاحظه زن بودنش را می‌کنم به حساب تشخص خودش و بی عرضگی من می‌گذارد. حتی یکی دو بار دست بلند کرده که بزندم.اما من همیشه دندان روی جگرم گذاشته‌ام و با خونسردی و گذشت نگذاشته‌ام کارمان به جاهای باریک بکشد.

بیچاره پدرم! از این چیزها هیچ خبری ندارد. یعنی خودم خبردارش نکردم. دلم نمی‌خواست خیال خوشش ر به هم بزنم. دلش خوش است که پسرش با عملگی درس خوانده و به جایی رسیده و زن و بچه دارد و خیلی خیلی خوشبخت است. همین امروز به دکتر گفت: «آخه آقای دکتر من یک قرون هم خرج درس خواندن این پسر نکردم. خودش رفت و خوند. هم کار می‌کرد وهم درس می‌خوند. حالا چرا باید هر چی داره خرج من کنه؟»

بعد روش را به طرف من کرد. توی چشم‌های ریز وگود رفته‌اش اشک جمع شده بود و گفت: «بابا الهی قربوند برم. بابا منا ببخش. بابا من چقدر در حق تو ظلم کردم. بابا تو اون سن و سالی که همه بچه‌های هم سن و سال تو تو کوچه بازی می‌کردن من تو را توی بیابان‌ها دوانده‌ام. بابا تو را به خدا کمتر خرج کن و بیش از این منا خجالت نده.»

از این حرف‌ها و حالت پدرم گریه‌ام گرفت. گرمی اشک را روی گونه‌هایم احساس کردم. از خودم بیزار و منزجر شدم. من خودم می‌دانم که آن طور که باید و شاید به پدرم کمک نکرده‌ام و ای کاش هزار یک بزرگواری او را من داشتم.

خیال می‌کند که من هم مادرم هستم، که سید یاسین که سید یاسین عمله طرق را ازیاد ببرم. خیال می‌کند آن روزهای بچگی از یادم رفته که با هم برای هیزم کنی به بیابان می‌رفتیم و ظهر که می‌شد لب یک چشمه زیر تک درختی که توی آن بیابان بزرگ بود می‌نشستیم و دوتا نان جو خشک را آب می‌زدیم که بخوریم. من آن روزها را خوب به یاد دارم درست مثل دیروز است، پدرم ان‌ها را دستمالی می‌کرد و هر جایس را که برشته و آب کشیده و نرم بود جلوی من می‌گذاشت و جاهای خشکش را که آب بر نمی‌داشت خودش سق می‌زد. من این جور محبت‌های بی ریای بابام را خیلی خیلی دیده‌ام. اما هیچ وقت این جور مواقع توی صورتم نگاه نمی‌کرد. که مثلا محبتش را به رخ من بکشد. من از سید یاسین انسانیت‌هایی دیده‌ام که هرگز نمی‌توانم فراموش کنم. برای اولین بار در عمرم از اینکه دنبال پول نرفته‌ام ناراحتم. اگر پول داشتم به هر وسیله‌ای که بود سید یاسین را به فرنگ می‌بردم تا اگر خوب هم نشود و بمیرد وجدانم راحت باشد که خرجش کردم. اگر چه سید با حرف و عملش همیشه دنبال پول رفتن را در نظر من یک عمل بد و پست جلوه داده است.

امروز بابام به دکتر گفت: «آقای دکتر من برا مردن آماده‌ام. شکر خدا چیزی باقی ندارم. پسرام بزرگ شده‌ان. زن و بچه دارن. دخترم شوهر کرده و بچه داره امروز حساب کردم دیدم من که بمیرم دوازده نفر از من باقی می‌مونن.»

بعد روش را به من گرداند و دستش را به طرف خدا (خدای بابام همیشه یا تو سقف است یا تو آسمان) دراز کرد و گفت: «الهی زندگیت از این هم بهتر بشه»!!!.

بیچاره پیرمرد. چقدر ساختن این خانه صد و دوازده متری خوشحالش کرده. هر دو سه ماه یک بار که برای دیدن بچه‌های من تو این خانه می‌آمد از دیدن این خشت و گل که مال پسرش بود حتی بیشتر از دیدن ما حظ می‌برد.

باز هم بیچاره پیرمرد. اگر واقعیت را بداند. یعنی بفهمد این خانه برای من مثل زندان است. آن وقت چه کار می‌کند؟ اگر بداند که جد و آبای من توی این خانه روزی صد دفعه زیر و رو می‌شوند. آن وقت آن کاخ خوشبختی که توی خیالش از این خانه برای من ساخته است خراب شود؟ کاش روی سر خود من خراب می‌شد تا از شر زنم خلاص شوم. شاید هم روی قلب پیر خودش خراب بشود. نه. من اصلا نمی‌خواهم. اما من خودم به هر حال هیچ وقت دل به این خانه نبسته‌ام و هرگز هم نخواهم بست.

پریروز زنم بی‌شرمی را به جایی رساند که می‌خواست با سیخ کباب بزندم. من هم برای اولین بار از جا در رفتم. مثل یک گنجشک بلندش کردم و بردمش بالای سرم. نزدیک بود که بکوبمش روی موزاییک‌های توی راهرو. اما یک هو عقلم سر جاش آمد. شیطان را لعنت کردم و باز صحیح و سالم گذاشتمش زمین و گفتم: «ببین تو این هفت سال هر وقت که دعوامون می‌شد می‌تونستم همچی کاری با تو بکنم اما دلم نمی‌خواست. از این به بعدشم می‌تونم ولی باز جلو خودمم می‌گیرم. برو حیا کن!» زنم ماتش برده بود. حتی یک کلمه هم حرف نزد. رنگ و روش شد مثل گچ دیوار. من بی اعتنا از خانه آمدم بیرون و رفتم بیمارستان پیش بابام. از حال بچه‌هام و زنم پرسید گفتم: «همشون خوبن و می‌خواستند بیان شما را ببینن اما چون مریضخونه است نخواستم که بچه‌ها بیان، زنم هم وایساد که بچه‌ها را بپاد. بابام گفت: «نه! نه! باباجون هیچ وقت نیذار که بچه‌ها بیان توی مریضخونه. اصلنم راضی نیستم که زنت بچه‌ها را ول کنه بیاد، الهی که خیر هم را ببینید. قدرش را دانسته باش که زن خوبی داریاگر گیر نی مثل مادرد می‌افتادی اون وقت می‌فهمیدی زن چه آفتی می‌تونه باشه.»

بیچاره بابام، خوشا به حالش، بی خبری چه دنیای خوبی است. نمی‌دانست که یک ساعت قبل چه اتفاقی افتاده بود و نمی‌داند شب که به خانه برگردم زنم رفته خانه باباش و نصف شب باباش تلفن می‌کنه که برای من هارت و پورت بکند. اما من با خونسردی به او می‌گویم » تو قاضی، هر کاری که بکنی قبول دارم.» و با همین حرف بادش را خالی می‌کنم.

بیچاره بابام نمی‌داند الان که دارم این یادداشت‌ها را می‌نویسم زنم به خانه برگشته است و الان خودش و خواهرش و بچه‌ها توی اتاق بزرگ روی تخت خواب‌ها راحت خوابیده‌اند اما من با یک تشک بی‌ملافه، تنهای تنها، مثل آدم‌های در به در توی اتاق کتابخانه مچاله شده‌ام. خودم هستم و خودم. هیچ کس را ندارم که براش درد دل کنم تا شاید یک ذره دلم خنک شود.

این راه حلی است که جناب ریش سفید محکمه یعنی پدر زنم - وهیات قضات - یعنی خانواده‌اش - فکر کرده‌اند. برای اینکه مردم نفهمند که من و دخترشان دعوا کرده‌ایم و با هم قهریم باید هر دو توی این خانه زندگی کنیم. اما این طوری! به عبارت دیگر این راه حل این بوده استکه من خرج خانم و بچه‌ها و خواهرش را بدهم. پول قسط خانه را هم که به اسم خانم است بدهم. پیش در و همسایه هم مترسک خانم و بچه‌ها و خواهرش باشم.

البته این را خودم خواستم که خانه به اسم زنم باشد. یعنی این کاری بود که پدرم به من یاد داد. او هم به وقتش خانه‌اش را به اسم مادرم کرده بود. به عقیده بابام برای مرد ممکن است هزار جور گرفتاری پیش بیاید. وقتی که پول و خانه در دست زن باشد اگر مرد گرفتار شد زن می‌تواند با بچه‌هایش زندگی کند. من از این کار پشیمان نیستم که هیچ خیلی هم خوشحالم. چون حداقل فایده این کار این است که آدم به یک جای به خصوص به اسم خانه دلبستگی پیدا نمی‌کند. هر چند که هم مادرم و هم زنم هر دوشان ثابت کرده‌اند که من و پدرم هر دومان اشتباه میکنیم و نه تنها مالک این خشت و گل نیستیم بلکه مثل آدم موظف باید همیشه دست و کیسه‌مان را هم پر کنیم تا این غربتی‌ه گرسنه نمانند. بله این دلبستگی نیست. ریش بستگس است.

بیچاره بابام این است سرنوشت پسرت که خیال می‌کنی زن درس خوانده و بچه‌های خوب و خانه و زندگی راحت دارپ.

اما من تا زنده‌ام هرگز این خیال خوش تو را به هم نمی‌زنم. همان طور که تا حالا پیشت تظاهر کرده‌ام باز هم تظاهر می‌کنم که با زنم خیلی خوبم و خوشبخت‌ترین آدم‌های روی زمینم. چه اشکالی دارد. من که نمی‌توانم مرض زندگی سوز تو را معالجه کنم، اقلا می‌توانم با تظاهر به خوشبختی در این کشاکش مرگ خیالت را از بابت اولادت که خودم باشم ناراحت نکنم.

حقیقتش این است که زنم هم تقصیری ندارد. من و او از یک طبقه نبودیم و زبانمان هم یکی نیست. در واقع او زن من نشد، زن لیسانس من شد. زن من نشد که با من زندگی بکند. زن یک جوان گدا گشنه شد که به سرش تسلط داشته باشد. حالا بیچاره عملا حسابش غلط از کار در آمده. من یابوی وحشی سرکشی هستم که هیچ کس نمی‌تواند زورکی سوارم شود. اصلا باباجان برای اینکه دلم را سبک کنم میخواهم برایت یک کاغذ بنویسم. اما برایت نمی‌فرستم. نگه می‌دارم پیش خودم. تو راحت باش و آسوده بمیر.

«سید جانم از خدا می‌خواهم که پیش از تو بمیرم و جان کندن و مردن تو را نبینم. با اینکه زنده بودن تو باعث می‌شود که من زجر وجود این زن را تحمل کنمو دم نزنم ولی باز هم می‌خواهم که سال‌های سال زنده بمانی. اما همین که تو از رنج این زندگی راحت شدی من هم از دست این زن خودم را راحت خواهم کرد. تو این را نخواهی شنید که من زنم را طلاق داده‌ام. تو از این قضیه ناراحت نخواهی شد.

اصلا چه من زودتر بمیرم و چه تو عاقبت ما در یک جایی به هم می‌رسیم. من به این موضوع اعتقاد دارم هر دو می‌رویم زیر خاک. مرا ببخش که به آن دنیا اعتقاد ندارم و به ریش هیچ کدام از این کلم به سرهای نورانی و بی نور، زنده و مرده تره خرد نمی‌کنم. اگر تو زودتر بمیری من این سوهان روح را حتما ول می‌کنم. تا وقتی که سر قبرت می‌آیم، روی خاکت بیفتم و های های گریه کنم، بی اینکه دلهره داشته باشم که الان زنم سر می‌رسد و فحشم می‌دهد که چرا لباس‌هایت را خاکی کرده‌ای!!!

یادت هست چقدر گیوه‌هایمان را زیر سرمان می‌گذاشتیم و روی خاک‌های کوچه و بیابان‌ها می‌خوابیدیم؟

یادت هست اون شبی را که جلو کاروانسرای میر پنج کاشان خوابیده بودیم و جعفر قهوه‌چی آمد تا با من «ور برود» من از خواب پریدم و نعره کشیدم؟ حتما یادت هست که چطور نزدیک بود او را بکشی.

یادت هست آن سال که از قم به تهران می‌آمدیم؟ با آن الاغ بیست و شش تومانی که توی حسین آباد میش مستها خریده بودیم، می‌آمدیم تهران که به قول تو طحافی یعنی «طوافی» کنیم. بین راه عزیز آباد و حسین آباد کناره گردگیر غربتی‌ها افتادیم. غربتی‌ها می‌خواستند لختمان کنند. اما تو زرنگی کردی و گفتی: «ما سیدیم ما رو به جدمون پیغمبر ببخشید.» با آنکه مدرکی نداشتیم که ثابت کنیم سید هستیم غربتی‌ها ما را لخت نکردند. حتی از ما معذرت هم خواستند. اما این غربتی‌ که الان با خواهرش و دو تا توله‌های من توی آن یکی اتاق خوابیده است با اینکه سواد دارد و شناسنامه‌ام را هم خوانده و توی قباله عقد خودش هم اسم من «سید رضا» نوشته است یک لحظه از آزار من فروگذار نمی‌کند. اما بابا یک وقت فکر نکن که زن من نفهم است. برعکس خیلی هم فهمیده است. ولی حقیقت این است که اجداد من و تو آدم‌هایی بودن که همدیگر را می‌دریدند. غارربتی‌ها ساده و خر بودند که آن روز از جد ما ترسیدند و ما را لخت نکردند. 

سید جان! الهی زبانم لال بشود. اگر راستش را بخوای من به جدم اعتقاد ندارم. ازش هم نمی‌خوام که شر زنم را از سرم کم کند. همین که تو از این دنیا رفتی خودم خودم را از دست این زن که شاید در تیپ و طبقه خودش زن خیلی خوبی باش، خلاص می‌کنم. بگذار دلش خوش باشد که من خانه و زندگی‌ام را برایش گذاشتم و رفتم.

سید جان! تو خسته‌ای. احوال نداری. بس کنم. خداحافظ اما امیدوارم که سال‌های سال از دست زنم راحت نشوم.

\chapter{چهارشنبه}

سه روز است که سید یاسین از بیمارستان رفته. برادرم او را به خانه برده است. امروز به دیدنش رفتم. تقریبا یک بعد از ظهر به آنجا رسیدیم. من بودم و دوستم. این اولین بار است که دوستی را به این خانه می‌برم. آخر خانه سید ساسین پشت دباغ خانه است. توی تمام جوی‌هایش لجن آبکی گندیده ایستاده است.

این خانه برای من خیلی عزیز است. تمامش پنجاه و پنج متر است. این خانه را پانزده سال پیش ساختیم. همان سالی که من به دانشسرا رفتم. زمینش را قسطی خریدیم. آن سال تابستان کار و بار من گرفته بود و و تقریبا دو هزار تومان کار کرده بودم. طوافی می‌کردم. یک الاغ داشتم. بابام هم یک الاغ داشت. بابام نمک می‌فروخت. فقط نمک. اما من از نمک فروشی بدم آمده بود از بس که بچه‌های کوچه تهرانچی مسخره‌ام کرده بودند. همین که صدام بلند می‌شد:

- «نمکیه! نمکه! کوبیده و نکوبیده نمکه!»

بچه‌ها دورم جمع می‌شدند و دنبالم می‌آمدند و دم می‌گرفتند:

- «نمکی آی نمکی یک درا بستس نمکی هفت درا نبستی نمکی.»

من لجم می‌گرفت. فحششان می‌دادم و با سیخونک به آن‌ها حمله می‌کردم. بچه‌ها ده بیست قدم فرار می‌کردند و دوباره برمی‌گشتند و همان شعر را دم می‌گرفتند. این بود که از نمک فروشی بیزار شدم و تصمیم گرفتم چیز دیگری بفروشم.

آن سال من «سیکل» گرفته بودم. باد «سیکل» توی کله‌ام بود. پیش خودم خیال می‌کردم که از همه این بچه‌ها بالاترم و بیشتر چیز می‌فهمم. شاید هم بچه‌ها حق داشتند این نمکی را مسخره کنند. به هر حال تصمیم گرفتم که نمک نفروشم. این بود که رفتم تو مایه فلفل فرنکی و لوبیا سبز. اتفاقا خیلی هم خوب کارم گرفت. یک بار یک گاله فلفل فرنگی که از صد کیلو هم بیشتر بود چکی خریدم به سه تومن و نیم. دو ساعت بعدش توی کوچه تهرانچی دو کیلویش را فروختم پنج تومان. خیلی خوشحال شدم. کشف تازه ای کردم. پس می‌شود توی تهران یک چیزی را پنجاه برابر قیمت اصلی فروخت. چون آن دو کیلو فلفل بیش از یک ریال برای من تمام نشده بود. خلاصه آن 
سال من دو هزار تومن پس انداز کردم. بابام هم تقریبا هفتصد هشتصد تومان کنار گذاشته بود. از این پول‌ هزار تومانش را دادیم پیش قسط و زمین همین خانه را خریدیم. به اسم ننه‌ام. هزار و دویست سیصد تومان هم دادیم تیر چوبی و حصیر خریدیم. سیصد چهارصد تومان هم پول بنا دادیم. برای گچ و آجر پول نماند. بنا شد با خشت بسازیم. و خشتش را هم خودمان بزنیم. شب‌ها که از کار برمی‌گشتیم با بیل و کلنگ می‌افتادیم به جان زمینمان و خاک ازش برمی‌داشتیم. از همین آب‌های لجنی که هنوز هم به مقدار کافی دور و بر هست می‌بستیم به خاک‌ها و گل می‌ساختیم. پاچه‌ها را ور می‌مالیدیم و با پای برهنه گل‌ها را ورز می‌دادیم. آن وقت من می‌رفتم می‌خوابیدم و بابام تا نزدیکی‌های صبح تمام این گل‌ها را با یک قالب خشت مالی که چهار تومان خریده بود، خشت می‌زد. غروب که از سر کار بر‌می‌گشتیم خشت‌ها نیمه خشک شده بود. آن وقت با بابام خشت‌ها را ور می‌چیدیم و زنجیروار به ردیف‌های مارپیچ روی پهلوهایشان وا می‌داشتیمتا زودتر خشک بشوند. آخرهای شهریور بود که یک بنا گرفتیم. یعنی بنا که چه عرض کنم یک شاگرد بنا. خودمان هم عملگی‌اش را کردیم و خانه ساخته شد.

مهرماه همان سال که من رفتم به دانشسرا و شبانه روزی شدم بابام و ننه‌ام و برادرهام و خواهرم از کاروانسرای حاج کدخدا اسماعیل آمدند توی این خانه و ما هم توی این دنیا مستغلات‌دار شدیم.

از مطلب دور افتادم. گفتم که یک بعد از ظهر من و دوستم رفتیم به خانه سید یاسین. دیروز که دیدمش فهمیدم که امروز باید بروم آنجا. آخر او باید وصیت می‌کرد. من پسر بزرگش هستم. دلش می‌خواهد راهی را که رفته است قطع نشود. یعنی که من ادامه بدهم. او به جاودانگی روح اعتقاد دارد. اما اعتقاد او این جور است که می‌گوید. روح منتقل می‌شود از پدر به پسر. همان طور که شیرینی یک زردآلو منتقل می‌شود از زردآلو به زردآلوی دیگر.

بابام بارها به من گفته است هرکاری که پدر و مادر بکنند به اولادشان برمی‌گردد.

به هر حال من می‌دانستم که بابام امروز وصیت می‌کند و راه آینده مرا به من نشان می‌دهد و روی این کارش حساب هم کرده است. به همین جهت می‌خواستم شاهدی داشته باشم. چون من نمی‌توانستم و نمی‌خواستم که قلم به دست بگیرم و آن چه را که پدرم می‌گوید بنویسم این کار دو عیب بزرگ داشت. اگر من کاغذ و قلم برمی‌داشتم خواهی نخواهی او را در این رنج می‌گذاشتم که موقع مردن است و وصیتنامه نوشتن. هر چند که بابام از مرگ ترسی ندارد. عیب بزرگترش این است که من می‌توانستم احیانا به وسوسه نفس اماره (این نفس اماره را برای اولین بار از بابام شنیدم و بعدها که سواددار شدم راجع به این کلمه خیلی چیزها خواندم.) تسلیم بشوم و وصیتنامه او را در حین نوشتن تغییر بدهم. این بود که با دوستم رفتم تا شخص ثالثی بین ما باشد و اگر روزی من راه غلط رفتم اقلا یک نفر از این همه آدمیان بداند که راه سید یاسین مستقیم بوده و این پسر نالایق اوست که «اریب» می‌رود. دست کم به خیال خودم اصلا 
دلم نمی‌خواهد «اریب» بروم.

آمدن این دوست یک حسن دیگر هم داشت و آن اینکه الان هم دیگر مجبور نیستم وصیتنامه سید یاسین را بنویسم. دوست من بوده و دیده و شنیده است. اگر لازم بود ما دو نفر می‌توانیم بشینیم و گفته‌های اورا روی کاغذ بیاوریم. حالا من می‌توانم این موضوع را که از ساعت یک تا تقریبا دو و نیم بعد از ظهر طول کشیده رها کنم و دنبال بقیه ماجرا بروم.

ساعت سه و نیم بعد از ظهر من و دوستم به تجریش آمدیم. با یک دیزی آبگوشت یک نفره که از قهوه‌خانه ممد نایینی گرفتیم هر دومان ناهار خوردیم و من دوباره به شهر برگشتم.

در راه یک چیز مخفی روی قلبم سنگینی می‌کرد. این چیز یک فکر بود. یک فکر مخفی.

وسط‌های راه، در عین ناراحتی از پرواز او و یا واضح‌تر بگویم از نزدیک بودن مرگ بابام خیلی خوشحال شدم و این فکر مخفی پیدا شد. «قبر بابای ن باید کجا باشد؟» فکر مخفی همین بود. قبر بابام باید نزدیک آن مراد و یا دست کم در همان قبرستانی باشده که قبر آن مراد نیز در آن است. مردی که بابام هر شب جمعه به یارتش می‌رفت و به «مرده‌پا»ها نمک می‌داد تا آب رو قبرش بپاشند.

خیلی خوشحال شدم که تکلیف من روشن شد. من باید به فکر خرید یک قبر باشم. قبری در ابن بابویه. قبری در کنار قبر آن مراد. مرادی که او را شهید می‌دانست. در آن قرن‌ها او را کشته‌اند و در چشم بابام او یک شهید است و هر شب جمعه به زیارتش می‌رود.

اما این خوشحالی زیاد طول نکشید. همین که به چهارراه پهلوی رسیدم و پیاده شدم و پیچیدم که به طرف اداره بیبیم غمی دیگر، غمی بزرگ و خیلی هم بزرگ دوباره قلبم را فشار داد. این یکی دیگر اصلا مخفی نبود. این غم، غمی بود که در خودم بود یعنی از وجود خودم برخاسته بود. غمی که فکر می‌کنم هر انسانی را خرد می‌کند. می‌فهمد که از مرحله انسانی به پایین افتاده و پست شده است. من آنقدر پست شده‌ام که می‌خواهم برای پدرم قبر بخرم. آیا این به آن معنی نیست که باطن من مرگ پدرم را می‌خواهد و می‌خواهم او را زنده به گور کنم؟

نزدیک بود که دیوانه بشوم. سر چهارراه کالج اگر یک لحظه دیرتر گفته پدرم به یادم افتاده بود شاید خودم را زیر ماشی انداخته بودم والان من اینجا نبودم و راننده اتوموبیل در زندان بود. بیچاره راننده.

بابام دیروز به من گفته بود که «می‌خواهم به وطنم برم.» و معنی حرفش این بود که وطن او دنیای دیگری است. من خودم را به کوچه علی چپ زدم و گفتم: «آقاجون مثل اینکه خیلی دلتون برای شهر تنگ شده! ایشالا بهتر که شدید با هم یک سفر می‌ریم.» اما او بی تامل این شعر را برایم خواند:
\begin{center}
«مرغ باغ ملکوتم نیم از عالم خاک

چند روزی قفسی ساخته اند از بدنم»
\end{center}

با همه این‌ها وقتی که به اداره رسیدم خیلی ناراحت بودم. خواستم بنسینم و به کار اداره برسم. دیدم ناراحتم و نمی‌توانم. چطور آدمی که احساس می‌کند که پست شده است می‌تواند بنشیند و چیز بنویسد. قلبم خیلی سنگینی می‌کرد. بلند شدم و بی اختیار به اتاقی که تلفن در آن هست آمدم. خانم ماشین نویس آنجا  نشسته بود و روزنامه توفیق می‌خواند. گوشی را برداشتم و روی دکمه قرمز زدم. تلفن بوق آزاد زد. صفر هشت را گرفتم. چند دقیقه طول کشید تا خانم متصدی صفر هشت گوشی را برداشت. گفتم: «خانم لطفا مطب آقای دکتر عبادی را بدید. توی خیابان شاه، پهلوی سینما چارلی.»

خانم گفت: «دکتر کی؟»

گفتم: «دکتر عبادی»

بعد از یکی دو دقیقه باز گفت:

- گفتید دکتر کی؟

- خانم گفتم دکتر عبادی، عین-ب-تشدید-الف-دال-ی

- آدرسشون

- عرض کردم، خیابون شاهِ کوچه پهلوی سینما چارلی. و لحظه‌ای بعد شماره را گرفتم و مردی از آن طرف گوشی را برداشت و گفت:

-الو

- جناب آقای دکتر عبادی!

- خود آقای دکتر را می‌خواید!

- بله

- گوشی خدمتتون باشه

به دکتر گفتم: «من پسر سید یاسین هستم و می‌خواهم بدانم احوال پدرم را که دیروز دیده‌اید واقعا چطور است؟» دکتر گفت: «شما روز آخر که پدرتون رو از بیمارستان بردند قرار شد که پیش من بیایید تا آخرین نتیجه‌ای را که از عکس‌برداری به دست اومده به شما بدم، اما نیومدید.»

- خوب حالا نتیجه را لطف کنید. من همین را می‌خوام.

- آقای مظلومی خیلی ببخشید ها، مایوس کننده است هیچ دوایی به درد نمی‌خوره. یکی دو ماه است که کبد از کار افتاده. رو کبد علامت سرطان هم دیده شده.

- پس می‌میره؟

- بله.

- تا کی؟

- یکی دو روز دیگه.

گفتم: «آقای دکتر ما مداوا (یعنی دوا خوراندن. این تنها موردی بود که معنی مداوا را کشف کردم.) را ادامه می‌دیم. دلم می‌خواد تا وقتی که هست عذاب نکشه. شما را به خدا امروز هم تشریف ببرید. خونه را که بلدید، برادرم همیشه اونجاست. برا ویزیتو دوا هم پول اونجا هست. لطفا دوایی بدید که درد نکشه. آخه من شنیدم که سرطان خیلی درد داره.

- درسه ما از لحاظ انسانی مداوا را ادامه می‌دیم.

گوشی را گذاشتم. دوباره دکمه قرمز را فشار دادم تا تلفن بوق آزاد زد. شماره دوستم را گرفتم و گفتم: «پدرم داره می‌میره. چطوره قبرش را نزدیک قبر همان مراد بگیریم؟»

- خیلی خیلی خوبه. اصلا باید همین کار را بکنیم. دیدی دیروز با چه حرارتی ازش حرف می‌زد. من که همچی ایمانی تو مردم این روزگار ندیده‌ام.

- پس تو برو دنبال پیدا کردن محل قبر! به هر قیمتی که باشه می‌خریم.

- بسیار خوب.

- منم می‌رم دنبال پول قبر.

خانم ماشین نویس که غیر از توفیق خواندن به حرف‌های من هم گوش می‌داد با تعجب بر و بر به من نگاه کرد. اما تا رفت حرفی بزند من شماره دیگری را گرفته بودم و از آن طرف سیم تلفنچی گفت: «شما آقای حسابدار هستین؟» گفتم: «نه من مظلومی هستم. سلام عرض می‌کنم . میخوام با آقای مجاهد صحبت کنم.» شاید هم من خیلی عجله کرده بودم. این اولین بار بودکه من از این تلفنچی چنین اشتباهی می‌شنیدم. آقای مجاهد رییس ماست. یعنی رییس اداره.

- خواهش می‌کنم آقای مظلومی گوشی خدمتتون.

و بعدش: «آقای مجاهد سلام! من احتیاج به حقوق یک ماهم دارم.» آقای مجاهد گفت: 

- خیلی خب بیا بگم بهت بدن. کی می‌خوای

- همین الان.

- خیلی خب بیا.

گوشی را گذاشتم. حالا رییس امور اداری با خانم ماشین نویس پچ پچ می‌کردند و متعجب بودند که اگر پدر من در حال مردن است، چرا من خونسردم و تازه خونسردی به کنار، چرا خوشحالم؟

اما من کاری به این کارها نداشتم. به دو از اداره بیرون دویدم و در عرض دو سه دقیقه از خیابان گذشتم و به ساختمان رییس نشین و قسمت حسابداری اداره ما، که دو سه تا کوچه و خیابان دورتر از قسمتی است که من در آن کار می‌کنم، رسیدم و یک راست رفتم به اتاق آقای مجاهد. در را باز کردم و گفتم: «سلام!» و خیلی عادی و شاید خوشحالتر از هر وقت بدون اینکه توضیحی بدهم و زمینه چینی کنم گفتم:
«بنویسید حقوق این ماه منو بم بدن می‌خوام قبر بخرم!» آقای مجاهد کمتر و خانمی که توی اتاقش نشسته بود خیلی بیشتر از این وضع من تعجب کردند. اما من به این جیزها کار نداشتم. من می‌خواستم برای بابام قبر بخرم و به پول احتیاج داشتم.

آقای مجاهد یادداشتی به آقای بدر نوشت که:

«مظلومی احتیاج فوری به پول دارد، برای خرید زمین مزار 

یادداشت را گرفتم و بدون خداحافظی از اتاق بیرون دویدم. نیم دقیقه بعد سه طبقه بالاتر توی اتاق بدر بودم. بدر داشت تلفن می‌کرد. بدون اینکه ملاحظه کنم که مشغول تلفن کردن است، یادداشت را دستش دادم. این شتابزدگی‌ها عادت همیشگی من است و من هرگز منتش را سر بابام نمی‌گذارم.

بدر از خواندن یادداشت کمی تکان خورد. اما نگذاشتم که معطل شود و زدم به مسخرگی و با خنده گفتم:

- آره دیگه من این پول را می‌خوام برای بابام قبر بخرم!

بدر خواست دلداری‌ام بدهد و چند مثال آورد ولی من با بی صبری منتظر پول بودم و باز با زبان بی حالی به او حالی کردم که برادر دلسوزی را بگذار برای بعد. فعلا من منتظر پولم و بس و هنوز هم بابام نمرده است.

گفت:

- آقای سمن پور حقوق آقای مظلومی را چک بکشید.

من با دستپاچگی وسط حرفش دویدم و گفتم:

- نه! نه! اگر هست نقد بم بدید!

دست آخر بدر از حساب بانک صادرات خودش که شعبه آن پهلوی اداره است هزار و بیست و هفت تومان چک کشید و من زیر یادداشت آقای مجاهد رسید دادم و دویدم پایین.

داشتم به در خروجی اداره می‌رسیدم که یادم آمد با آقای مجاهد خداحافظی نکرده‌ام. برگشتم و رفتم و در اتاق مجاهد را یک هو باز کردم و گفتم:

- خیلی ممنونم، پول رسید.

ضمنا باز چشمم به همان خانم و یک آقای دیگر افتاد و از اینکه در نزده بودم و سلام نکرده بودم کمی خجل شدم و با یکه خوردگی به آن آقا گفتم: «سلام!»

مجاهد همراه با آن پوزخند مخصوص خودش گفت:

- حالا بابات مرده؟

گفتم:

- نه نمرده! اما وصیت کرده براش قبر بخرم.

خیلی دلم می‌خواست که آن دو نفر توی اتاق نبودند تا به آقای مجاهد می‌گفتم: «می‌خوام هر طوری شده نزدیک قبر مراد برا بابام قبر بخرم.»

مجاهد از خوشحالی و وضع من تعجب کرد. هر چند که از من خیلی از این رفتارهای عوضی دیده است. چشم‌های عقابی‌اش را به من دوخت و با قیافه ظاهرا بی‌تفاوت همیشگی‌اش که کمی تفاوت و تعجب در آن خوانده می‌شد گفت:

- عجله نکن.

طفلکی مجاهد با همه زیرکی‌اش از کجا بداند که خوشحالی من از چه قماش است. از تمام پول‌ها و حقوق‌هایی که از فرهنگ و اداره و مدرسه‌ها و اشخاص مختلف گرفته‌ام فقط همین هزار و بیست و هفت تومانی که او به این سادگی نوشت و به من دادند مرا این طور خوشحال کرد. زیرا برای اولین بار بابام به من احتیاج پیدا کرده بود و من می‌خواستم که احتیاج او را رفع کنم. گر چه حقیقت این است که این بار هم آسید یاسین پیرمرد نمکی سر پل سیمان به به کسی احتیاج ندارد و اظهار احتیاج هم نکرده است و من خودم این احتیاج را درک کرده‌ام. شاید هم این نوعی خودخواهی باشد که من این مساله را این طور تعبیر کرده‌ام و این صحنه‌ها را به وجود آورده‌ام تا خودم را راضی کنم که نمردم و یک کار برای بابام انجام دادم.

\chapter{بیست و سوم مهر ماه ۱۳۴۵}

- اونجا چه خبره؟

- داریم عدس می‌پزیم. از باباتون چه خبر؟

- خوب شد!

این‌ها جمله‌هایی بود که بین من و خواهر زنم رد و بدل شد. جریان این است که ساعت یازده بعد از ظهر (بیست و پنج دقیقه پیش) که به خانه رسیدم دیدم چراغ‌های راهرو و آشپزخانه روشن است. وقتی که در را باز کردم دیدم زنم و خواهرش توی آشپزخانه هستند. اول خواستم محلشان نگذارم و بروم ولی بعد پرس و جویی کردم و معلوم شد که عدس می‌پزند تا برای قابلمه پسرهام که برای ناهار به مدرسه می‌برند عدس پلو درست کنند.

دیشب ساعت ده و نیم برای سومین بار به دیدن بابام رفتم. حالش خیلی بد بود. خیلی ناراحت و بد نفس می‌کشید. صورتش گار کشیده بود. چال و چوله‌های صورتش کاملا معلوم بود. اما وضع صورت بندی و ریشش حالتی روحانی به او داده بود.

چند دقیقه نشستم. چند نفری هم توی اتاق بودند. به من گفتند که بالای سر بابام بنشینم. من هم همین کار را کردم. بعد سید مهدی، سید علی محمد هم به برادرم احمد که پشت بابام نشسته بود و بابام به او تکیه داده بود گفت:

- حالا این طوری درسه؟

برادرم کفت:

- خب چکارش کنم؟

- پاشو بخوابونیمش.

سید مهدی این را گفت و خودش بلند شد و با کمک چند نفر دیگر متکا را از پشت برادرم برداشت و پهلوی دیوار اتاق گذاشت.

من نمی‌فهمیدم می‌خواهند چه کار بکنند. همان طور که نشسته بودم خودم را یک کمی جا به جا کردم. خیال می‌کردم که می‌خواهند بابام را بخوابانند، که راحت تر نفس بکشد. هرچقدر هم که سید مهدی چپ چپ نگاهم کرد باز ملتفت نشدم. توی خیالات خودم بودم و به حال خراب بابام فکر می‌کردم. عاقبت به زبان آمدند که: «بلند شو!» من هم بلند شدم چند نفر سر تشک را گرفتند و بابام را رو به قبله خواباندند. تازه فهمیدم که چه خبر است. خیلی دلم می‌خواست که بتوانم گریه کنم اما نتوانستم. با اینکه می‌دانستم که بابام مردنی است، هیچ دلم نمی‌خواست که این وضعیت را ببینم.

به هر حال بابام را رو به قبله خواباندند. همین که پاش را رو به قبله کردند و لحاف را روش انداختند به هوش آمد و چشم‌هاش را باز کرد. من توی درگاه اتاق ایستاده بودم. بابام من را دید. چشم‌های ریزش را ریزتر کرد و به من براق شد. من از نگاه بابم یک حالی شدم. یک حالی که نمی‌توانم بنویسم چطور بود. همین قدر می‌دانم که بغض بیخ گلویم را گرفته بود. اول خیال می‌کردم که بابام با نگاهش التماس می‌کند. التماس می‌کند و می‌گوید که: «بابا رضا جون، دارم می‌میرم، یک کاری بکن که نمیرم.» و از اینکه هیچ کاری از دستم بر نمی‌آمد درمانده بودم. در حینی که نگاه بابام را این طوری تعبیر و تفسیر می‌کردم، بابام دست‌هایش را از زیر لحاف در آورد و به طرف من دراز کرد و گفت:

- «بابا رضا جون باز اومدی خجالتم بدی! آخه چرا کار و زندگیت را ول می‌کنی و میای اینجا؟ بابا جون برو دنبال کارت. مگه قرار نبود که بری مسافرت، برو بابا جون من هم امروز و فردا می‌رم. بابا زود باش برو، من از روی تو خجالت می‌کشم.»

گفتم:

- آقا باز که از این حرف‌ها زدید! شما خیلی حق به گردن من دارید. من هیچی نتونستم و نمی‌توانم برا شما بکنم. خیل خوب حالا که می‌خواید برم، می‌رم. «خداحافظ شما!»

بابام هم با صدای کشیده‌ای گفت:

- خدا... حافظ... شما... باشه... 
 
بابام با من خداحافظی کرد وچشمش را به هم گذاشت. آن وقت همان طور که چشمش به هم بود لگن خواست. احمد لگن آورد. بابام گفت: «بخوابونیدم رو دنده چپم.» همین کار را کردند. سرش را آورد لب متکا و گفت: «لگن را بگذارید زیر دهنم.» همین که لگن زیر دهنش گذاشته شد شروع کرد به استفراغ کردن. استفراغش خوان غلیظ و سیاه رنگی بود که مثل شیری که بچه شیری بالا آورده باشد، دلمه دلمه شده بود. استفراغش که تمام شد، چشمهایش را باز کرد و چند لحظه به خون دلمه شده‌ای که توی لگن می‌لرزید نگاه کرد. بعد با لبخندی حاکی از رضایت که بسیار هم غم انگیز بود سرش را آهسته آهسته تکان داد. و به وسط متکا برد و همچنان که لبخند غم انگیزش را به لب داشت دوباره چشم‌هایش را به هم گذاشت. بعد کم کم لبخندش محو شد و سرش شبیه کله‌های مومیایی مصری‌ها که در کتاب‌های تاریخ می‌کشند شد. من دیگر نتوانستم و یا نخواستم آنجا بمانم. به برادر کوچکم محمود گفتم:

- بیا من و تو بریم. چون بابا نمی‌خواد ما اینجا بمونیم.

این را می‌توانستم بگویم، چون چند دقیقه قبل به من گفته بود: «برو پی کارت» محمود را هم باید می‌بردم چون طفلکی از شش صبح تا هفت بعد از ظهر سنگ تراشیده بود. از هفت و نیم تا نه و نیم هم رفته بود آموزشگاه و درس خوانده بود و حالا خیلی خسته بود و از حال رفته بود. از این گذشته دیدن حال بابام خیلی منقلبش می‌کرد.

دلم می‌خواهد راجع به محمود دو سه کلمه بیشتر بنویسم. امسال بیست و پنج سالش تمام می‌شود. پانزده سال پیش با مادرم و احمد و خواهرم سکینه به تهران آمدند. من و بابام تهران بودیم. پنج شش ماه اولش را دسته جمعی در کاروانسرای حاج کدخدا توی یک اتاق زندگی می‌کردیم. اتاقش بزرگ بود. خودمان جلو اتاق می‌خوابیدیم و ته اتاق هم الاغ‌هامان را می‌بستیم. بعد که این خانه را ساختیم و آمدیم توش مستقر شدیم، بابام گفت: «محمود باید یک صنعت دستی مستقل یاد بگیرد.» این بود که محمود را گذاشت پیش استاد حبیب‌اله سنگتراش. محمود الان یک سنگتراش خیلی خوب است. در ضمن اینکه می‌رفت سنگتراشی بابام تشویقش کرد که درس بخواند. شش سال پیش یعنی وقتی که نوزده سالش بود شروع به یاد گرفتن الفبا کرد. الان توی آموزشگاه آذر مشغول خواندن کلاس پنجم ریاضی است. بابام محمود را خیلی دوست داشت. من هیچ وقت ندیدم که از او گله‌ای داشته باشد. به عکس احمد که بابام را خیلی می‌چزاند و بابام همیشه از او شکایت داشت.

با محمود از در خانه بیرون آمدیم. شوهر خواهرم دنبالمان دوید و گفت: «فکری برای جواز دفن بکنیم.»

گفتم: «خیلی خوب من و محمود می‌ریم خانه دکتر عبادی و می‌گیریم. با برادرم آمدیم تا میدان شوش و از آنجا تاکسی گرفتیم. به راننده تاکسی گفتم: « داداش برو سینما چارلی.»

در تمام راه من و برادرم با هم حرف می‌زدیم. خلاصه حرف هر دوی ما این بود که بابای ما خیلی خیلی خوب بابایی بود. اما حیف که دارد می‌میرد و ما هیچ کاری نمی‌توانیم براش بکنیم.

راننده تاکسی هاج و واج شده بود. چون ما ظاهرا می‌رفتیم سینما اما حرف‌هامان مربوط به مرگ و میر بود. حوصله نداشتم براش توضیح بدم که داداش ما سینما نمی‌ریم. می‌ریم پیش دکتری که خونه‌اش پهلوی سینما چارلیست تا جواز دفن بابامون را، که رو به قبله است بگیریم. قبل از اینکه من سر حوصله بیام از تاکسی پیاده شده بودیم و در خانه دکتر را می‌زدیم. اما دکتر خانه نبود و ما برگشتیم. به محمود گفتم: «حالا که خونه دکتر را یاد گرفتی اگه خدای نکرده طوری شد، فردا صبح زود بیا و جواز دفن را بگیر.»

از سینما سعدی تا چهارراه مخبرالدوله با برادرم قدم زنان آمدیم و حرف زدیم. هیچ وقت من و محمود این قدر با هم نزدیک نبوده‌ایم و حرف نزده‌ایم. اگر احساسات و حرف‌های ما دو تا برادر را طی این پرسه کوتاه بتوانم بنویسم خودش یک کتاب بزرگ می‌شود. اما من که کتاب نویس نیستم. سر چهارراه مخبرالدوه از هم جدا شدیم. از محمود خواهش کردم اگر بابامان طوری شد به من تلفن کند. نصف شب به خانه رسیدم. دوشاخه تلفن توی راهرو را کشیدم تا فقط تلفنی که توی کتابخانه بالای سر خودم هست زنگ بزند. با آنکه یکی دو ساعت جان به سر بودم تلفن زنگ نزد. پس بابام زنده است. اگر مرده بود محمود تلفن می‌کرد. هر طوری بود خودم را خواباندم.

ساعت چهار از خواب پریدم. انگار خواب آشفته دیده بودم. دیگر خوابم نبرد تا تقریبا ساعت پنج. آن وقت باز یک چرت خوابم برد. اما توی خواب با بابام بودم.

- درینگ! «فقط یک بار»

خوب آلود گوشی را برداشتم. زنگ زنگ تلفن بود. یادم است که احتیاط کردم ضمن برداشتن گوشی دستم به بابام نخورد. بعد چشمم را باز کردم و دیگر از بابام خبری نبود. از آن سر سیم شوهر خواهرم حرف می‌زد. گفت:

- آقای مظلومی خیلی خیلی ببخشید. من نتوانستم دوستتون را پیدا کنم. هر چی زنگ زدم تلفنش جواب نداد این بود که مجبور شدم شما را خبر کنم. چند روز پیش به شوهر خواهرم گفته بودم که چون ممکن است به مسافرت بروم اگر بابام طوری شد به خانه تلفن نکنند. به دوستم تلفن کنند و شماره کتابفروشی او را داده بودم.

- خوب تمام کرد؟

- بله

- کی؟

- نزدیکای ساعت چهار.

- من الان میام.

به ساعتم نگاه کردم. شش و پنج دقیقه بود. لباسم را پوشیدم و از اتق بیرون آمدم. خواهر زنم که از حرف زدن من بیدار شده بود و توی راهرو بود گفت:

- چی چی شده‌ای؟

گفتم:

- هیچی رییسم بود، تلفن کرد که امروز زودتر به اداره برم!

شش و ربع در خانه دوستم بودم. زنگ زدم. دوستم با زیر پیراهن و شلوار پشت در آمد. گفتم:

- زود باش بریم بابام مرد.

- بیا تو تا من دست و صورتم را بشورم.

- خیلی خوب زود باش.

ساعت هفت در خانه دکتر را زدیم. خانمی در را باز کرد. رفتیم توی اتاق انتظار دکتر. خانم گفت: «الان دکتر میاد. شما مریض هستید؟»

گفتم: «نه آمده‌ایم جواز دفن یکی از مریض‌های آقای دکتر را بگیریم.

خانم به ارمنی صدا زد: «فردریک............

و ما صدای دکتر را شنیدیم که به زبان فارسی و خطاب به من در حالی که از پله‌ها بالا می‌آمد گفت:

- یک ساعت قبل برادرتون جواز دفن را گرفت و رفت.

دکتر توی راهرو رسیده بود و ما هم از اتاق بیرون آمدیم و با هم روبرو شدیم. با دکتر سلام و احوال پرسی کردیم. من چشمم را توی چشم دکتر اندختم و گفتم:

- آخرش مرد!

دکتر گفت:

- من که به شما گفتم که این سرطان لامسب....

هفت و نیم در خانه سید یاسین بودیم. مردم جمع بودند و منتظر ماشین «مرده کش» بودند.

خیلی‌ها بام احوال پرسی کردند. رفتم توی خانه. توی اتاقی که بابم خوابیده بود. بابام ساکت و آرام، بدون یک ذره حرکت، زیر یک لحاف رو به قبله خوابیده بود.

دوستم را صداش کردم. آمد. میرزا ابراهیم و میرزا علی اصغر هم پایین پای بابام نشسته بودند و قیافه گریان به خودشان گرفته بودند. از گریه میرزا ابراهیم خنده‌ام گرفت. اما هر طور بود خودم را نگه داشتم. به دوستم گفتم:

- ببین چطور آرام خوابیده!

او گفت:

- حرف‌هایش را زده.

من و دوستم ساکت و آرام مدتی این منظره را نگاه کردیم و با هم حرف زدیم. بدون آنکه یک کلمه از دهانمان بیرون بیاید. بعد هر دو از اتاق بیرون آمدیم. هنوز ماشین مرده کشی نرسیده بود. با همان ماشینی که از شمیران آمده بودیم (ساعتی کرایه کرده بودم) به مسگر آباد رفتیم. شوهر خواهرم و پسر عمویم هم آنجا بودند. با آنکه بابام شناسنامه نداشت دم مرده خورها را دیده و جواز دفن گرفته بودند. ماشین مرده کسی که نوبتش بود هنوز آماده نبود. یعنی بنزین نداشت. ماشین نوبت بعد هم خراب بود. عاقبت ده تومان دادیم به راننده‌ای که ماشین خودش برای تعمیر توی گاراژ بود تا با یکی از ماشین‌هایی که آنجا بی شوفر ایستاده بود همراه ما بیاید. راننده رفت پشت یک ماشین قراضه نشست که رشن کند. پسر عموم یک ماشین مرده کش شورلت بی دماغ نو را که ایستاده بود نشان داد و گفت:

- داداش اگر با این بیای ده تومن انعام می‌دم.

یارو فورا پرید پایین و رفت پشت مرده کش لوکس نشست. اینجا هم من خنده‌ام گرفت. حتی نتوانستم جلوی خنده‌ام را بگیرم. چشمم توی چشم دوستم افتاد. او هم خنده‌اش گرفت. گفتم:

- ببین پسر عموم چقد با معرفته! می‌خواد بابامو تو ماشین آخرین مدل روانه قبرستون کنه.

بالاخره خودمان با ماشین کراه از جلو و شوهر خواهرم و پسر عمویم و ماشین مرده کش لوکسشان به دنبال ما راه افتادیم.

آمدیم تا سر خیابان دباغخانه. آنجا ما ایستادیم و آن‌ها جلو افتادند. چون دوستم می‌خواست به برادرش تلفن کند که پول به حسابش بگذارد که چکش برنگردد. بعد از تلفن ما هم دنبال آن‌ها رفتیم. نزدیک خانه بابام عباس زنده علی و چند نفر دیگر جلو آمدند و تابوت را از توی ماشین مرده کش برداشتند و اصرار کردند که جنازه را تا سر خیابان شاه عبدالعظیم رو دوش بیاورند. تا رفتم سرم را بخارانم دیدم جنازه روی دست مردم به سرعت می‌دود. چند دقیقه بعد جنازه و مردم توی خیابان شاه عبدالعظیم نزدیکی‌های مدرسه شاه بودند. با اصرار و التماس توانستیم جنازه را توی ماشین مرده کش بگذاریم. چون رگ «سید دوستی» جمعیت به جوش آمده بود و می‌خواستند جنازه «آقا نمکی» را تا ابن بابویه روی دوش ببرند. یک ربع بعد یعنی تقریبا ساعت هشت و ربع صبح جنازه بابام به در قبرستان ابن بابویه رسید.

(امشب خیلی تقلا کردم که بخوابم. درست مثل بابام که دیشب تقلا می‌کرد که بمیرد. حالا این یادداشت‌ها را قطع می‌کنم. شاید خوابم ببرد. اگر بتوانم بخوابم فردا صبح دنبالش را می‌نویسم و اگر نشد که همین امشب.)

حالا ساعت چهار و ربع صبح است. نیم ساعت پیش تز خواب پریدم. خواب می‌دیدم که سر قبر بابام بودم. داشتند گل روش می‌ریختند. یک مشت گلش روی من ریخت. خودم را پس کشیدم و دستم را جلو صورتم آوردم که گل روم نریزد. دستم به دیوار اتاق خورد و از خواب پریدم. اول یک حالت منگی بی‌تفاوتی داشتم. مثل ماری که زمستان از زیر خاک درش بیاورند. نمی‌توانستم فکر کنم. برای اینکه ببینم خوابم یا بیدار، و راستش را بخواهید، برای اینکه ببینم زنده هستم یا مرده آهسته خودم را تکان دادم. دیدم زنده هستم. خواب هم نیستم، بیدارم چون نوری را که از پنجره توی اتاق تابیده بود، تشخیص می‌دادم و غیر از این مهمترین علامت زنده بودنم را، یعنی سوزشی را که سال‌هاست زیر آخرین دنده چپ قفسه سینه‌ام وجود دارد، حس می‌کردم. از همه این‌ها گذشته جریان صبح دیروز را هم یادم بود.

دیروز هم همین ساعت‌ها بیدار بودم. اما دیروز تا امروز خیلی فرق دارد. شاید دیروز این موقع بابام مرده بود، اما من امید داشتم که زنده است. امروز این امید را ندارم. می‌دانم که مرده است. حتی می‌دانم که این یادداشت‌ها را باید از کجا ادامه بدهم:

گفتم که جنازه بابام به ابن بابویه رسیده بود. بابام همیشه سر پل سیمان و اطراف ابن بابویه نمك می‌فروخت. هر شب جمعه هم همان طور که قبلا اشاره کرده‌ام سر خرش را کج می‌کرد و می‌رفت ابن بابویه. این کار بابام هم فال بود و هم تماشا، هم زیارت اهل قبور می‌کرد و هم به «مرده پاها» نمك می‌فروخت و هم ....

خودش برام تعریف کرده بود که يك شب جمعه که باخرش توی ابن بابویه رفته بوده يك آدم خیلی خوش لباس و عصا قورت داده به بابام گفته بود:

- مرتیکه چرا با الاغ توی ابن بابویه اومده‌ای.

بابام هم بدون اینکه معطلش بکند جواب داده بود:

- اولا که مرتیکه خودتی. ثانيا هم از کجا که پیش خدا خر من از تو بالاتر نباشه. تازه اگر خدا را هم تو کار نیاریم و تو لباساتو در بیاری خرمن را بدون پالون صد صد و پنجاه تومن می‌خرند در صورتی که تو را یك قرون هم نمی‌خرند. مرد حسابی مردم به قبرستون میان که عبرت بگیرن و تکبر نکنن تو تو قبرستون اومدی که جانشین فرعون بشی؟

این را که گفته بود. آدم خوش پوش عارش شده بود که خودش جواب بابام را بدهد چند تا از مرده پاهای مقبره‌ها را خواسته بود که بابام را از قبرستان بیرون کنند اما مرده پاها گفته بودند:

- آقا «نمکی» توی این قبرستان آزاد است و هیچ کس حق ندارد که جلو او و خرش را بگیرد.

آن وقت آدم خوش پوش مثل عقرب تف انداخته دمش را روی کولش گذاشته و از قبرستان بیرون رفته بود.

اینها را نوشتم تا بدانید که بابام توی ابن بابویه غريب نبود.

دیروز در تمام طول راه، از در مدرسه شاه، که تابوت بابام راتوی نعش کش گذاشتند تا در قبرستان ابن بابویه که تابوت را از توی نعش کش در آوردند من با خودم فکر می‌کردم و می‌خندیدم. اگر دوستم که پهلوم نشسته بود می‌پرسید:

- چرا می‌خندی؟

می‌گفتم:

- برا این می‌خندم که نمردم و دیدم که عاقبت برای يك دفعه هم که شده بابام این راه را با ماشین اومد، آن هم با ماشین آخرین سیستم و البته به لطف پسر عمو. تا يك ماه پیش بابام هر روز این راه را پای پیاده به دنبال الاغش که نمك بارش بود می‌آمد.

جنازه بابام که به در ابن بابویه رسید، پیاده شدم و دویدم پیش قاسم آقا که همه کاره قبرها و مرده شوخانه ابن بابویه است و گفتم:

- خوب داداش «سید نمکی» را آوردیم دیگه باقیش با خودت.

روز پنجشنبه که با شوهر خواهرم آمده بودیم زمین قبر بخریم همین که قاسم آقا شنید که برای سید نمکی می‌خواهیم رنگ روش تغییر کرد و گفت: «لا اله الا الله» بعد اصرار کرد که: «سید خیلی حق به گردن ماها داره. شما زمین نخرید. اولندش که ایشااله بهتر می‌شه. دومندش به چشم من يك جای خیلی خوب براش در نظر می‌گیرم. اگه خدای نکرده مرد شما فقط برسونیدش اینجاو کاریتون نباشه.»

تا قاسم آقا رفت «مرده بر»‌هاش را خبر کند جنازه
بابام روی دست مردمی که از تهران با جنازه آمده بودند وارد صحن شد و قاسم آقا هاج و واج پشت سر آنها می‌دوید که بیاید و در «مرده شوخانه » اش را باز کند.

قاسم آقا که در را باز کرد، جماعت تابوت را بلند کردند و پای سنگ «مرده شویی» گذاشتند. این تقريبا دویست نفر مردمی که امروز دنبال جنازه سید یاسین آمده اند همين يك روز بیکاری در شام شبشان واقعأ تأثیر دارد.
بابام را لاي يك پتو پیچیده بودند. پتو راهم لای يك قالیچه. قالیچه را توی تابوت گذاشته بودند. روی تابوت هم يك پارچه سبز خیلی خوش رنگی که گویا شال سید مهدی سید علی محمد بود کشیده بودند.

توی مرده شوخانه اول سبز را از روی تابوت برداشتند. بعد قالیچه را باز کردند. وقتی که قالیچه را باز می‌کردند سید تقی حاج سید رضی پدر زن احمد مثل سگ‌های قصابخانه که منتظرند تا تکه ای از لاشه گوسفند را دور بیندازند تا آنها بقاپند، کمین کرده و منتظر قاپیدن قاليچه بود که کسی ندزدد. در طول مدتی که مشغول بيرون آوردن جنازه بابام از توی تابوت بودند سیدتقی با دستپاچگی و انتظار تکرار می‌کرد:

- قالیچه چی را بدید، قالیچه چی را بدید، قاليچه چی را بدید!!!

تا آنکه عاقبت سید تقی پرید و «قالیچه چی» را قاپید بعدش هم پتو را. همین که سید تقی پتو را هم قاپید چشمم به مرده بابام افتاد. بابام چشم‌هاش را روی هم گذاشته بود. دهنش هم بسته بود. معلوم بود که دندان‌های مصنوعی اش توی دهنش نبود حتما اگر بابام ریش نداشت قیافه اش خیلی مضحك می‌شد. اما ریش بابام باعث شده بود که مردنش با خواب بودنش فرقی نداشته باشد . بابام همیشه می‌گفت: «ريش جزو محاسن مرد است.» وقتی که در باره ریش پیغمبر و امامها حرف می‌زد کلمه ریش را به کار نمی‌برد بلکه مثلا می‌گفت: «محاسن حضرت سید الشهدا پر از خون شد» یا اینکه «حسن و حسین سوار پیغمبر می‌شدند و محاسن حضرت را می‌گرفتند و در عالم بچگی شتر سواری می‌کردند و از این حرف‌ها.....

تنها جایی که من به «محاسن» یعنی خوبی‌های ریش بابام معتقد شدم همین جا بود. بابام خواب بود و به خلاف همیشه اصلا خرناسه نمی‌کشید. هر قدر هم که این ور و آن ورش می‌انداختند بیدار نمی‌شد که هیچ، اصلا خم به
ابروش نمی‌آورد.

بچه که بودم - يك جای دیگر هم انگار گفتم - با بابام می‌رفتیم هیزم کنی. صلات ظهر بعد از آنکه نانمان را می‌خوردیم زیر تك درخت توی بیابان می‌خوابیدیم. بابام خیلی زود خوابش می‌برد. اما من یا خوابم نمی‌برد یا اگر خوابم می‌برد، خیلی زود از خواب می‌پریدم. آن وقت بنا می‌کردم به شمردن خرناسه‌های بابام. بعضی وقت‌ها هم به وصله‌های پیراهن و شلوارش خیره می‌شدم و کوک‌هاش را می‌شمردم.

گاه گاهی که خرناسه بابام قطع می‌شد وحشت برم می‌داشت و نفسم بند می‌آمد. همین که خرناسه دوباره شروع می‌شد من هم نفس راحتی می‌کشیدم. علامت زنده بودن بابام برای من همان خرناسه کشیدنش بود. وقتی که خرناسه نمی‌کشید، می‌ترسیدم که مرده باشد. اگر چند دقیقه خرناسه نمی‌کشید دستپاچه می‌شدم و مثلا يك خار توی پوست پشت دستش فرو می‌کردم. آن وقت بابام از خواب می‌پرید و کتکم می‌زد و من مثل کسی که بزرگترین گناه را کرده باشد شرمنده می‌شدم. نفسم در نمی‌آمد. هیچ وقت به بابام نگفتم چرا این کار را می‌کنم و بابام لابد آن کارهای مرا کارهای بچگانه حساب می‌کرد. هیچ وقت نفهمید که من از مردن او همیشه وحشت داشتم و هیچ وقت دلم نمی‌خواسته است که بمیرد.

توی «مرده شوخانه» هم خیلی دلم می‌خواست بروم يك چيز نوك تيز به پشت دست بابام فرو کنم تا شاید خرناسه‌اش بلند بشود اما دیدم اگر این کار را بکنم همه خیال می‌کنند که دیوانه هستم.

مرده شو لباس کهنه‌های بابام را از تنش در آورد. پیراهن کرباسی بابام يقه عربی بود یعنی چاك جلو پیراهنش تا جلو سینه‌اش بیشتر نبود. درست همان قدر که به راحتی پیراهنش توی سرش برود. اما مرده شو پیراهن بابام را از سرش بیرون نیاورد، آن را چاك داد. باز خنده‌ام گرفت. برای اولین بار بود که می‌دیدم بابام يك پیراهن جلو باز دارد که جلوش مثل پیراهن مردم کراواتی باز است، اگر چه يقه شکاری ندارد. دو سه دفعه آمدم به یارو بگویم بابام از این فرم پیراهن خوشش نمی‌آید اما باز ترسیدم دیوانگی خودم را ثابت کنم.

مرده شو بابام را لخت مادرزاد کرد و به پشت روی سنگی مرده شوخانه خواباند من تا به حال خوابیدن کسی را روی سنگ مرده شوخانه ندیده بودم. این هم چیز خنده داری است. یک هو یاد نفرین‌های مادرم افتادم.

هروقت مادرم با من یا بابام دعواش می‌شد می‌گفت:

- «الاهی روسنگی مرده شو خونه‌ات بخابونن» و من تا حالا معنی این نفرین را به این روشنی درك نكرده بودم.

خلاصه بابام را لخت و عریان روی سنگی مرده شوخانه خواباندند و آب رویش ریختند. همین که سطل آب را رو سرش ریختند در موهای ریشش فرق وا شد، و صورتش مثل جمجمه‌هایی که روی جعبه دواهای سمی‌دواخانه‌ها می‌کشند در نظرم آمد.

اول دلم می‌خواست وقتی که بابام می‌میرد مسافرت باشم. این آرزو به این جهت بود که نمی‌خواستم مرده بابام را ببینم تا همیشه قیافه‌اش برام زنده باشد. اما حالا که مقدر شده بود که من اینجا باشم و جنازه بابام جلو چشم من شسته بشود، پس چه بهتر که همه چیزش را ببینم. همین طور هم شد. جاهایی از تن بابام را دیدم که هیچ وقت ندیده بودم.

من با بابام خیلی به حمام رفته بودم اما توی حمام لنگ دور کمرش می‌بست و هیچ وقت نمی‌شد «آنجاها» یش را دید. من آن روز «آنجاها»ی او را هم دیدم.
بیچاره پدرم! و بی حیا من! بیچاره پدرم که حتی نمی‌توانست عورت خودش را
بپوشاند. و بی حیا من که حتی به عورت بابام هم نگاه کردم و آن را دیدم و ورانداز کردم.

بله من امروز از فرق سر تا ناخن پای پدرم را ورانداز کردم.

افسوس که من نقاشی بلد نیستم والا، همین گوشه کاغذ، با چندتا خط، ساده، جنازه بابام را در روی سنگ مرده شوخانه نقاشی می‌کردم.

واقعا حیف که نقاشی بلد نیستم. این نقاشی که آدم با کلمات می‌کند اصلا احساس آدم را بیان نمی‌کند. خیال می‌کنم اگر می‌توانستم با خط، خط ساده حرفم را بزنم، بهتر می‌توانستم خودم را خالی کنم. طفلك بابام. باز هم نمی‌توانم خدمتی بهش بکنم. وضع سر بابام هیچ فرقی نکرده بود. فقط چون دندان‌های مصنوعی‌اش را از دهنش بیرون آورده بودند پوزه‌اش باريك شده بود و لب بالایی اش روی لب پایینی‌اش فشرده شده بود. اما ریشش، به قول خودش محاسنش، به هر حال این معایب را می‌پوشاند. از لوله راست بینی‌اش هم خون بیرون زده بود و به اندازه دوسه سانتیمتر روی گونه راستش دویده بود و خشك شده بود.

حالا بیایم سر تنه اش:

خیال کنید که تابستان است. يك زن دهاتی چند تا ترکه باريك درخت انار چیده و برگ و تیغ آن را کنده و ترکه‌های صاف را به شکل قفسه سینه پهلوی هم گذاشته است. روی این ترکه‌ها يك پارچه ململ نازك زرد رنگ انداخته است و برای مشك دوغش سر پوش درست کرده است. البته پارچه ململ را محکم نکشیده و در فاصله ترکه‌ها دره‌های کوچکی درست شده است طوری که می‌توانیم ترکه‌های زیر ململ را بشماریم.

زیر این سرپوش مشك دوغ بدون باد را گذاشته است که توی آن مقداری دوغ هست و به کوچکترین تکانی که به مشك بدهیم دوغ‌های توی آن تلوتلو می‌خورد و چون سر پوش به اندازه تمام مشك دوغ نیست نیمی از مشك از زیر سرپوش بیرون افتاده است و تلوتلو خوردن دوغ را از نیمی از مشك که از سرپوش بيرون است به راحتی می‌توان دید. تنه جنازه بابام عينأ همین حالت را داشت:

 یعنی استخوانهای سینه‌اش عین ترکه‌ها و پوست روی استخوان‌های سینه عین پارچه ململ و شکمش عين قسمت بیرون از سر پوش مشك دوغی بود که گفتم. هر وقت هم که مرده شو تكانش می‌داد آب‌های توی شکمش مثل دوغی که توی مشك بی باد باشد تلوتلو می‌خورد.

بیان حالت دست‌ها و پاها و گردنش خیلی ساده است: 

چندتا نی باريك را بردارید و آنها را به اندازه طول گردن و دو تا دست و دو تا پای يك آدم کوتاه قد خیلی خیلی لاغر ببرید. خوب حالا پنج قطعه نی دارید. روی هر يك از این پنج قطعه نی از همان پارچه ململ زرد رنگ بکشید و کوتاهترین قطعه نی را بین سر و تنه قرار بدهید، دو تا از نی‌ها را که متناسب با دست باشد بین کف دست و شانه و دوتای دیگر بین قوزك پا و پایین تنه. والسلام. اما چرا گفتم نی انتخاب کنید؟ چون همانطور که نی بند بند است و بین بندهای آن برآمدگی هست کشکک زانو و آرنج و مهره‌های پشت گردن بابام هم برآمدگی داشت.

مرده شو مثل قصاب‌هایی که چهار دست و پای گوسفند را می‌بندند و آن را می‌خوابانند تا تکان نخورد و آن وقت با خیال راحت پشت چاقوشان را دم دهنشان می‌گذارند و با لبشان نگه می‌دارند و آستین‌هاشان را بالا می‌زنند و سر فرصت چاقو را به دست می‌گیرند و سر حیوان را می‌برند، يك این طور حالت‌هایی داشت. خیلی خونسرد بود. وقتی که بابام را لخت عور کرد و مطمئن شد که تکان نمی‌خورد، انگشتر عقیقش را از
دستش در آورد و روی رف لب حوض مرده شوخانه گذاشت. ساعتش را هم از جیب جلیقه‌اش در آورد و پهلوی انگشترش گذاشت. آن وقت آستین‌هایش را بالا زد و مشغول کار شد. بی انصاف لعنتی اول دست چپ بابام را گرفت. و انگشت‌های دست را که گره شده بود غرغ و غرغ شکست تا باز بشود. وقتی که صدای غرغ و غرغ انگشت‌های دست بابام بگوشم می‌رسید من در جا می‌لرزیدم و توی انگشت‌هام احساس درد بسیار شدیدی می‌کردم. نزديك بود که از شدت درد فریاد بکشم اما یک هو یادم آمد که اگر این کار را بکنم مردم می‌فهمند که دیوانه‌ام.

تمام انگشتهای دست چپ بابام گره شده بود، اما دست راستش وضع دیگری داشت. انگشت اشاره دست راستش سیخ بود و چهار انگشت دیگرش خوابیده. درست حالت دست‌های آدمی را داشت که در حال سخنرانی به مردم اشاره می‌کند.

همه آدمهایی که آنجا ایستاده بودند به بابام نگاه می‌کردند و از اینکه این همه لاغر شده بود تعجب می کردند. حتما اگر يك دانشجوی دانشکده پزشکی آنجا بود دلش برای این اسکلت بی دردسر که زحمت
از هم سواکردن گوشت و استخوان را نداشت آب می‌شد.

مرده شو روی عورت بابام يك تكه كرباس سفید انداخت و داشت بدنش را کيسه و صابون می‌مالید که من را صدا زدند. اول خیال کردم می‌خواهند بيرونم ببرند که بابام را نبینم و به همین جهت اعتنا نکردم و بیرون نرفتم. اما بعد دیدم قاسم آقاست بیرون رفتم. قاسم آقا گفت: «بیا ببین جایی را که براش در نظر گرفته‌ام خوبه یا نه» قاسم آقا جلو افتاد و من به دنبالش، رفتم تا رسیدیم به در مسجد یکی از کله گنده‌ها و مقبراش. قاسم آقا جایی را که درست رو به روی در مسجد بود نشان داد و گفت:

- اینجا خوب جائیه، همه مردمی که برای نماز و فاتحه خونی توی مسجد میرن باید از روی این قبر رد بشن و ثوابش به روح این سید اولاد پیغمبر می‌رسد. اما نمیشه که دور قبر نرده کشید. اصلا نرده کشیدن دور قبر کار خوبی یم نیس. به جد خود آقا نمکی قسم که دورور قبر مراد جا پیدا نمی‌شه. اگر نه رو چشمم اونجا بش جا می‌دادم. من می‌دونم که «آقا نمکی» چقد به مراد احترام می‌ذاشت. هر شب جمعه می‌اومد سر قبرش و فاتحه می‌خوند. اصلا می‌دونی قسمت «آقا نمکی» اینجا بود. به جدش قسم که من این قبر را چون جلو در مسجد بود نگهداشتم که هزار تومن بفروشم اما سید خیلی حق به گردن همه ما داره با شما همان دویست و بیست تومنی که باید به مالك قبرسون بدم و قبر‌های معمولی را می‌فروشم حساب می‌کنم.

گفتم:

- اشکالی نداره.

وقتی قاسم آقا گفت «قسمت» آقا نمکی «اینجا بوده» یاد مرده چال کردن توی ده خودمان افتادم... در ده ما معمول است که وقتی کسی می‌میرد و می‌خواهند برایش قبر بکنند قبر کن به قبرستان می‌رود و کلنگش را به بالا پرتاب می‌کند. کلنگ چرخ زنان پایین می‌آید و هر جا که سر کلنگ به زمین فرو رفت آنجا قبر آن مرده است. اگر سه بار کلنگ را بیندازند و فرو نرود از آن جای قبرستان به جای دیگری می‌روند و آنقدر این کار را تکرار می‌کنند تا اینکه نوك کلنگ به جایی فرو رود و قسمت مرده و خانه آخرتش معلوم شود. اما در قبرستان‌های تهران قبرها را هم مثل خانه‌ها، تو هم تو هم می‌سازند. فقط يك ديواره ده پانزده سانتیمتری آجری قبرها را از هم جدا می‌کند. خلاصه تنگی و گشادی خانه آخرت بستگی دارد به کیسه صاحب مرده. و قسمت مرده کارهای نیست مگر اینکه ملك نقاله کار مرده‌ها را بسازد و آنها را جا به جا کند. بله قسمت بابای ما هم این بود که به لطف قاسم آقا جلو مسجد خاکش کنند.

قاسم آقا یکی از قبرکن‌هاش را صدا کرد و گفت:

- عباس آقا این قبر را برا «آقا نمکی» خالی کن. قبر کن یکه خورد و گفت:

- چی؟ مگه آقا نمکی مرد؟ قاسم آقا که اشك تو چشمش جمع شده بود گفت:

- آره مرد!

قبر کن گفت: «لااله الالله» و مشغول کارش شد.

من و قاسم آقا برگشتیم به طرف مرده‌شوخانه. توی مرده‌شوخانه مرده‌شو داشت پاهای بابام راسنگ پا می‌زد. کف و لبه‌های کف پای بابام مثل زمینی رسی که آب زیادی توش ببندند و دیگر مدتها آب به آن ندهند تا خشك خشك بشود ترك خورده و قاچ قاچ بود. این ترک‌ها را من موقع زنده بودن بابام دیده بودم. خود با بام عقیده داشت که «باد و برنگ» بدنش از این ترک‌ها بیرون می‌رود. اما من می‌دانم علت این ترک‌ها چی بود. بابام در عمرش هرگز جوراب پاش نکرده بود. بیشتر تابستان‌ها تقریبا پا برهنه راه می‌رفت چون که همیشه گيوه‌اش پاره پوره بود. زمستان‌ها هم پای لختش را توی گیوه یا يك جفت کفش کهنه می‌کرد. این بود که همیشه پاش ترك ترك بود. انگشت‌های پای بابام رو هم سوار شده بود . روی پاهاش هم هنوز باد داشت.

کار شستن بابام تمام شد. چند تا سطل آب روش ریختند تا کف‌های صابون از تنش برود. هر دفعه که آب می‌ریختند دستمال کرباسی از روی عورتش پس می‌رفت و بیشتر مردمی که آنجا ایستاده بودند، مخصوصا بچه‌ها، بیش از هر جای دیگر به عورتش خیره می‌شدند. کوچکترین برادرم محمد هم آنجا ایستاده بود و با چشم‌های ورقلنبیده و گردش خیره خیره به جنازه بابام نگاه می‌کرد و هر وقت که دستمال از روی عورتش پس می‌رفت می‌خندید. این بچه را بابام تو تهران درست کرده بود.

بعد از این کارها بابام را سه دفعه دیگر غسل دادند يك دفعه با آب سدر. يك دفعه با آب کافور و يك دفعه دیگر هم با آب خالص.
بوی کافور که توی مرده‌شوخانه پخش شده قاسم آقا هم کار بریدن و درست و راست کردن خلعت با بام را تمام
کرد و آن را روی تخت آجری مرده‌شوخانه پهن کرده بود. توی ده ما به کفن خلعت می‌گویند و معمول است که هر کس به کربلا یا مشهد می‌رود برای خودش خلعت می‌خرد و يك شب روی ضریح حضرت می‌اندازد و تبرك می‌کند و می‌آورد. اما با بام با اینکه چندین مرتبه به کربلا و مشهد رفته بود چنین کاری نکرده بود. این بود وقتی که مرد هیچی برا خودش نداشت، حتی خلعت.
بابام را از روی سنگ مرده‌شوخانه بلند کردند و روی تخت مرده‌شوخانه روی خلعت چلواری که قاسم آقا براش پهن کرده بود خواباندند. در اینجا هم خندیدم. شمردم و دیدم که یک بار با بام پارچه چلواری تنش کرد. آخر پیراهن‌های بابام همیشه کرباسی بود.

مرده‌شو اول توی حلق بابام يك لوله پنبه چپاند. بعد هم يك تكه بزرگ پنبه را توی دهنش چپاند. نصف پنبه از دهن بابام بیرون ماند و روی ریشش را گرفت. توی سوراخ‌های بینی و گوشش هم باز پنبه چپاند. دو تا تکه پنبه هم روی تخم چشم‌هایش گذاشت و بعد سر بابام را با يك باريکه چلوار که شبیه عمامه آخوندها بود پیچید.

من از این کار‌های مرده‌شو خیلی بدم آمد. دلم می‌خواست بیخ گلوی این مرده‌شوی بیشعور را بگیرم و سرش داد بزنم که: «پدرسوخته احمق! چرا توی حلق بابای من پنبه چپوندی! و چرا صورت نازنینش را به این ریخت لعنتی در آوردی؟» اما دیدم اگر این کار را بکنم همه می‌فهمند که دیوانه‌ام و از این گذشته فوری کافر می‌شم. چون لابد این کارها را باید با مرده بکنن تا روز قیامت که بلند می‌شود و توی صحرای محشر می‌رود از دور بشناسندش و بفهمند که مسلمان است و پای علم نصر و من الله و فتح و قريب جایش بدهند.

بعد از آنکه سر بابام را پیچید. یک تکه چلوار را مثل لنگ دور کمر و کپلش پیچید. اما قد این لنگ خیلی کوتاهتر از لنگ معمولی بود که توی حمام می‌بندند. تا بالای زانوی بابام بود. شبیه مینی ژوب خانمها بود. از اینکه دیدم مینی ژوب تن بابام کردند نزديك بود قهقه بخندم و همه بفهمند که من دیوانه‌ام.

قاسم آقا از توی يك كاغذ مچاله چند تا انگشتر تربت در آورد و به دست مرده‌شو داد. مرده‌شو یکی از انگشترها را به یکی از انگشت‌های دست راست بابام کرد. چند تا کار دیگر هم با بابام کرد که یادم نیست. آنوقت کفنش را به جسدش پیچید و بالا و پایین کفنش و دور شکم و دور زانوهای بابام را از روی کفن با يك كناره چلوار بست و بعد قالیچه چی را از دست سید تقی حاجی سید رضی که آنجا ایستاده بود و تماشا می‌کرد گرفتند و توی تابوت پهن کردند. و بابای سفید پوش من را توی تابوت خواباندند و پرهای قالیچه را برگرداند. و از نو بابام لای قالیچه ناپدید شد. چون سید مهدی سید علی محمدآن دور و بر نبود شال سبز رنگ و رو رفته میرزا علی اکبر را هم گرفتند و روی تابوت کشیدند و تابوت بابام را به دوش گرفتند و «به عزت و شرف لا اله الا الله» گویان از مرده‌شوخانه بیرونش بردند و جلو مقبره ابن بابویه زمینش گذاشتند تا نمازش بخوانند.

نماز که تمام شد، دوباره تابوت را بلند کردند و آوردند تا سر قبر، هنوز کار قبر کن تمام نشده بود. داشت ته قبر را صاف و صوف می‌کرد.
کار قبر کن که تمام شد جنازه را همان طور که توی قالیچه پیچیده بود از توی تابوت در آوردند و لب قبر گذاشتند. پرهای قالیچه را پس زدند و قبر کن با دسته کج بیلش قد بابام را همانطور که توی کفن پیچیده بود اندازه گرفت. قد بابام تقريبا يك وجب کمتر از دو برابر دسته کج بيل قبر کن بود. حالت رضایت صورت قبرکن نشان می‌داد که قبر اندازه است. آنوقت قبرکن که هنوز لبخند رضایت توی صورت گوشت آلودش بود خاک‌های ته قبر را با دستش جمع کرد و در بالای لحد کپه کرد. قبر کن از قاسم آقا چند تا ده شاهی برای کناره گرفت. آن وقت اشاره کرد که مرده را پایین بدهند. لحظه ای بعد بابام به دنده راست توی قبر خوابیده بود. به این ترتیب که سرش روی آن کپه خاك بود یعنی به طرف مغرب، پاهایش به طرف مشرق، روش به طرف جنوب و پشتش به طرف شمال. درست به عکس ترتیب نقشه‌های جغرافيا.

قبرکن که سبیل‌های زرد و چشم‌های درشتش بیش از لباس پاره پوره و قد کوتاهش جلب توجه می‌کرد با دست‌های گوشت آلود و پنجه‌های کلفتش گره بالای سر کفن بابام را باز کرد و آن پارچه عمامه را هم باز کرد و مثل تحت الحنك آخوندها دور گردن با بام انداخت.

بناگوش چپ بابام را روی کپه خاك گذاشت. پنبه توی دهنش را هم در آورد و روی سینه اش گذاشت. اما آن پنبه حلقش را در نیاورد. خیلی دلم می‌خواست که پنبه توی حلقش را هم در بیاورد. اما در نیاورد و من هم صدا از گلوم در نیامد. و نگفتم که این کار را بکند. ولی پنبه توی سوراخ بینی اش را در آورد.
در تمام این مدت من بالای سر قبر با بام جلو در مسجد
ایستاده بودم و نگاه می‌کردم. از مردم هم غافل نبودم که با تعجب به من نگاه می‌کردند و از قیافه‌شان فهمیده می‌شد که از من بدشان آمده است که این طوری بی تفاوت و بدون اینکه گریه و زاری کنم دارم تو خاك گذاشتن بابام را تماشا می‌کنم. اما من به این کارها کاری نداشتم می‌خواستم حالا که مرده پدرم را دیدم تا آخرش را ببینم. ملا مرتضی آخوند ساده و کودنی که دوست بابام بود و بابام همیشه او را به خاطر همین سادگی و کودنی‌اش دوست داشت و می‌گفت: اگر ملا مرتضی کودنه اقلا این حسن را داره که مثل دیگران حقه باز نیست. بالای قبر نشست که تلقين بخواند. قبر کن يك آجر بزرگی ختایی بالای سر بابام روی لحد گذاشت و تا سينه بابام از چشم پنهان شد، دلم ریش ریش شد. نه به این علت که بابام مرده است. بیشتر به خاطر این بود که دهن بابام مثل دهن ماهی‌های بزرگ جنوب که تو شیراز دیده بودم باز مانده بود و قبر کن هم هیچ توجهی به آن نکرد و همانطور بازماند. من هم صدام در نیامد و نگفتم که: «دهنش را ببند».

دهان بابام حالت آدمهایی را داشت که دهنشان را باز می‌کنند يك چیزی بگویند اما نمی‌توانند و یا جرأت نمی‌کنند که بگویند. هنوز هم این حالت بابام جلو نظرم است. شاید هم هیچ وقت از یادم نرود.

سید تقی حاجی سید رضی در حالی که محكم قالیچه عزیزش را در بغل می‌فشرد به قبرکن گفت:

- وقتی که آقا تلقین می‌خونن دسدا ( یعنی دستت را) به بغل گوش مید بیذار.

اما قبرکن که با بی اعتنایی به او نگاه می‌کرد بند روی شکم کفن را باز کرد و يك آجر ختایی دیگر هم پهلوی آجر اولی گذاشت طوری که من دیگر از کفل به بالای بابام را نمی‌دیدم. سید تقی باز حرفش را تکرار کرد و اصرار کرد که:

- یالا دسدا بغل گوش مید بیذار تا تلقينا بشنوه. و قبرکن دستش را الکی زیر آجر به طرف لحد برد و خنده به من فشار آورد. چون می‌دیدم که دست قبرکن اگر به جایی از جنازه باشد به مچ دستش است یا در ... اما هر طوری بود خودم را نگه داشتم و نخندیدم تا مردم نگویند که من دیوانه‌ام.

عاقبت تلقين تمام شد و قبر کن بندهای روی زانو و پایین پای کفن را هم باز کرد و با دو تا آجر دیگر روی لحد را پوشاند و گل آهکی را که قبلا درست کرده بودند روی آن ریختند و چند دقیقه بعد قبر بابام هم سطح زمین بود و مردم اطراف آن نشسته بودند و انگشت‌های دست راستشان را توی خاك فرو کرده بودند، و فاتحه می‌خواندند.

و من همان طور که ایستاده بودم قبر برام مثل شیشه شده بود و خیلی واضح دهن باز مانده بابام را می‌دیدم که مثل دهن ماهی‌های بزرگ جنوب که توی شیراز دیده بودم باز بود و هیچ کس به آن توجهی نکرد. حالا در نیاوردن پنبه توی حلقش سرش را بخورد چرا این قبر کن بی شعور دهن بابام را نبست که این خاطره برای همیشه از بابام توی ذهن من بماند.

\begin{thebibliography}{99}
\bibitem{}
شماره ثبت در کتابخانه ملی ۱۴۲۷ به تاریخ ۵۳/۱۰/۱۵
\bibitem{}
شرکت سهامی کتاب‌های جیبی
\end{thebibliography}

\end{document}
